\documentclass[a4paper]{report}
% Use swiss german letters
\usepackage[utf8]{inputenc}
% Language: english
\usepackage[english]{babel}
% Blindtext package
\usepackage{blindtext}
% Fancy Figures
\usepackage{graphicx}
% Display the Bibliography in the TOC
\usepackage{tocbibind}
% Better lists
\usepackage{enumitem}
% Use biblatex
\usepackage[style=apa,backend=biber,citestyle=authoryear]{biblatex} 
% Define the bibliography file
\addbibresource{bibliography.bib}
% To let LaTeX handle "
\usepackage[autostyle, english = british]{csquotes}
\DeclareLanguageMapping{english}{english-apa}
% To have text wrap around pictures
\usepackage{wrapfig}

% Titlepage
\newcommand*{\titleAP}{\begingroup % Create the command for including the title page in the document
	\centering
	\vspace*{\baselineskip} % Whitespace at the top of the page
	
	{\Large Pascal Baumann} and {\Large Nathanael Gogniat} and {\Large Reto Scheidegger}\\[0.167\textheight] % Author name
	
	{\Huge\bfseries Intellectual property in the digital age}\\[\baselineskip]
	
	{\Large \textit{Thesis paper InfKol FS2017}}\\
	\today
	
	\vspace*{3\baselineskip} % Whitespace at the bottom of the page
	\endgroup}

\graphicspath{{./img/}}

\begin{document}

\titleAP

\begin{abstract}
	\blindtext
	% Pascals writes this
	
\end{abstract}
\newpage

\tableofcontents

\newpage

\chapter{Introduction}
\label{ch:Intro}
% Nati writes this

\section{Digital Age}
\label{sec:Digital age}
The term "Digital Age" is a description of the current period. The Cambridge Dictionary describes it as "the present time, when most information is in a digital form, especially when compared to the time when computers were not used". \parencite{CambridgeUniversityPress2014}

The definition found in Wikipedia goes a bit further by incorporating the business impact and the article starts as following: "The Information Age (also known as the Computer Age, Digital Age, or New Media Age) is a period in human history characterized by the shift from traditional industry that the Industrial Revolution brought through industrialization, to an economy based on information computerization \parencite{WikiInfoAge2017}

\subsection{Influence of the Digital Age on IP}
The digital availability and the instant accessibility of the information (any time from anywhere) is in stark contrast to the availability of information in earlier ages. This becomes obvious when looking at the example of books as means of information distribution in Europe.

During the period of antiquity, information was manually written on tablets, scrolls and similar media. Thus, saving information was a time consuming and burdensome task.  Later on, books with pages of parchment became the norm. These books were manually copied by monks or other scholars. As a consequence, these items were rare and expensive. Later on, the single pages of a book were distributed to different copyists to increase the speed of copying. Although this parallelisation of copying made the process of copying a book faster, the effort used was only distributed and not minimised; thus, books remained rare and expensive. This began to change, after Johannes Gutenberg invented the printing press with movable type around 1450. This development sped up the process of copying books. Because, after the set up of the print plate was completed, the same page could be printed multiple times in a timely manner. However, copying still required the presence of a physical specimen to copy from. With an increasing number of affordable books, the knowledge contained could be spread further. But this process of distribution was still hindered by long, slow and dangerous transports. While the advancement in printing and transportation sped up the process of copying and distribution later on; instant access to information was still not available, because the information was bound to a certain geographical location. Before the rise of the internet, people still had to go to libraries for research purposes, often-time being required to order specimens of the books from other libraries - or order a personal copy of a book.

With the establishment of the internet as a commodity -- available not only from home and the office, but from nearly everywhere via mobile devices -- the point of origin of a published text no longer matters for the global availability. The distribution of information no longer happens in concentric regions around a physical specimen, but happens instantly, everywhere, and at the same time. Getting a copy of a book no longer requires burdensome printing and transportation, but can be done easily by doing some mouse clicks. While this can be beneficial to the human race as a whole, the originator of the books (now in digital form) is no longer sure that he will be compensated, as a book might be distributed freely using file sharing as soon as one copy has been sold.

\section{Copyright}
\label{sec:Copyright}

\subsection{Definition and scope of copyright}
\label{sec:CopyDef}
Copyright is a form of intellectual property right generally associated with works of artistic value. It is intended to fight plagiarism. The legal term copyright describes the rights a creator holds over their literary and artistic work. Such works include books, paintings, movies, computer programs, databases, advertisements and maps. Copyright usually only applies to a practical expression and not ideas, procedures or mode of operation. \parencite{CopyGov}

Copyright includes economical and moral rights. The economical rights allow the rights holder to derive financial reward from the use of protected works. The holder of economic rights can authorize or prohibit reproduction, broadcast, public performance, recording, translation and adaptation of protected works. Moral rights protect non-economic interests of the creator like being credited for their work. \parencite{WikiCopy}

According to the Berne Convention copyright is obtained automatically with the act of creation and usually covers a very long time span. \parencite{WikiBerne} There is no registration needed before one can claim and enforce copyright over their own works. Often voluntary registration is available nonetheless to help solve disputes over ownership. Such registrations can help facilitate financial transactions related to a copyright and assignment or transfer of rights to other parties. There exist a number of organizations offering collective management of copyrights, exercising those rights and related transactions on behalf of the rights owner.

\subsection{Copyright on the internet}
Many users of the internet have a number of misconceptions about copyright and how it applies to their usage of internet services. The most important misconceptions are outlined in an article by Vorys at al. \parencite{Vorys2013} and summarized here.

Probably the most common misunderstanding is that everything found on the internet can be freely downloaded and used as the basis of own works. However this is not true in most cases. As mentioned in chapter \ref{sec:CopyDef} copyright is applied automatically to all eligible works and unlicensed reuse is prohibited unless explicitly authorized by the rights owner. The source of works and whether you payed for it or not -- for instance an iTunes purchase versus a YouTube download -- is not relevant.

Another very common misconception often seen on YouTube is that it suffices to either give credit to the original creator, add a copyright notice or claim so called fair use. \parencite{Vorys2013} Adding a copyright notice or crediting the original author has no legal meaning by itself. Proper authorization to use the work is still required. An exception to this can be works published under a Creative Commons license as described in chapter \ref{sec:CC}.

The aforementioned fair use is also a source for confusion among internet users. Fair use is a doctrine that under certain circumstances allows the use of copyrighted material without acquiring permission from the rights holder. It is intended solely as a defence in a lawsuit and can not be regarded as a safe harbour. Whether fair use is applicable has to be decided by a court on a case by case basis (\cite{WikiFair} and \cite{Vorys2013}). As such claiming the usage of copyrighted material as fair use -- for instance in the description text of a YouTube video -- is meaningless.

The legal practice around fair use could however change in the future due to a ruling of the United States Court of Appeals for the Ninth Circuit from 2015 (affirming an earlier decision by the US District Court for the Northern District of California from 2008) in the case Lenz vs. Universal Music Corp. In this case the court reinforced that \textquotedblleft the fair use of a copyrighted work ... is not an infringement of copyright\textquotedblright\ and decided that \textquotedblleft in order for a copyright owner to proceed under the DMCA ... the owner must evaluate whether the material makes fair use of the copyright\textquotedblright\ before sending a take down notice. (\cite{LenzUniversal} and \cite{WikiLenzUniversal})

\section{Patents}
\label{sec:Patents}
According to Wikipedia, a patent is "a set of exclusive rights granted by a sovereign state to an inventor or assignee for a limited period of time in exchange for detailed public disclosure of an invention. [...]  Patents are a form of intellectual property." \parencite{WikiPatent2017}

Patents are thus used to encourage disclosure of new invention - thus making it available for others to spur further developments while at the same time granting the patent-holder privileges to cash in on the invention. The idea behind it can e.g. be found in the Copyright and Patent clause, Article I, Section 8, Clause 8 of the United States Constitution: "The Congress shall have Power [...] To promote the Progress of Science and useful Arts, by securing for limited Times to Authors and Inventors the exclusive Right to their respective Writings and Discoveries [...]" \parencite{Washington1787}.

It is important to note that the patent holder was never intended to be the sole profiteer of the patent deal, instead creation of patents was intended to be a "quid pro quo" deal. On the one hand, the patent holder got a limited monopoly while the public, on the other hand, got the invention in a way that would make it usable and, after the monopoly has ended, reproducible. This was established in various court opinions and orders, even as early as 1858, when Judge Daniels wrote in Kendall v. Winsor (62 .S. 322): 

"It is undeniably true, that the limited and temporary monopoly granted to inventors was never designed for their exclusive profit or advantage; the benefit to the public or community at large was another and doubtless the primary object in granting and securing that monopoly." \parencite{Curtis1858}.

One should take note, that in his opinion, the primary goal was the benefit to the public, not the benefit of the patent holder. Another court opinion defined the exact benefit to the public in 1945 in SCOTT PAPER CO. v. MARCALUS MFG. CO.:

"By the patent laws Congress has given to the inventor opportunity to secure the material rewards for his invention for a limited time, on condition that he make full disclosure for the benefit of the public of the manner of making and using the invention, and that upon the expiration of the patent the public be left free to use the invention." \parencite{Stone1945} Justice Stone thus stressed that patents are merely a compensation for making and using the invention. This goes diametrically against so called NPE, Non-Practising Entities - or, in other words, companies buying or creating patents without plans on using these patents in their products, but abusing them to get license payments from practising entities wanting to incorporate the same ideas in their own products.

\section{Trademark}
\label{sec:Trademark}
In legal terms, a trademark is globally accepted as \textquotedblleft a protected sign which is used to distinguish the products or services of one business from another\textquotedblright \parencite{SwissFederalInstituteofIntellectualProperty2017}.
Commonly, a trademark concerns the logo of a business, but it can also be a slogan or an animation (for example the roaring MGM lion).
Trademarks in itself are not a protected commodity just by their existence (they are in the USA; and are protected by common law, but a party has a stronger case when defending a registered trademark, and may use the \textregistered). They are only protected after registering the trademark, otherwise an opposing party using the same trademark has a much stronger claim to the same mark. \parencite{StatesPatent2016}

\section{Notable Organisations}
\label{sec:Orgas}

\subsection{BSA}
\label{ssec:BSA}
The Software Alliance (changed its name from \underline{B}usiness \underline{S}oftware \underline{A}lliance) is an advocate group representing the interests of the software industry in countries around the world. It was established in 1988 and has its headquarters in Washington DC. Notable members include, amongst others, Adobe, Apple, Autodesk, IBM, Microsoft, Oracle, and Siemens. The current CEO, Victoria Espinel, studied to become a telecommunications lawyer before being appointed as advisor on intellectual property issues to President Obama in 2009 and her subsequent appointment to her current position inside the BSA. \parencite{Rogers2017}
\vspace{\baselineskip}

\noindent The BSA's mission is based on three overriding policies:
\begin{itemize}
	\item Safeguarding intellectual property rights and protections
	\item Opening global markets to digital trade
	\item Enabling the growth of cloud 
\end{itemize}
\parencite{BSA2017}


To reach this stated goal, the Software Alliance believes that patents, copyrights, trademarks, and other IP rights should be comprehensive and enforceable. Trading partners, whether on a global or local scale, should not use their legislation to favour local industries or discourage international competition. And lastly, companies should not abuse their intellectual property to create unfair competitive advantage.

The BSA concentrated its effort for pushing the cloud in pushing for specific policies advancing the cloud, forging bilateral or multilateral agreements that ease the movement of data across borders. It promotes the use of sound data protection practises to expand and fortify the public trust in digital commerce. But even these data protection policies should leave enough leeway for technological innovation and new services, especially cloud computing.

The BSA garnered critic for aggressively enforcing the licence rights of its members under the guise of "audits" and providing bounties for people sending in hints of unlicensed software use in companies. Due to high penalties for unlicensed software and the need for very extensive proof of ownership, many small and medium sized businesses failed to provide ample documentation and were subsequently fined; some of these businesses even to the point of bankruptcy. These actions garnered the BSA the label to be akin to a bully and not the shining knight for IT innovation as presented in their mission statement. \parencite{Gaskin2009}

\subsection{SIIA}
\label{ssec:SIIA}
Similar to the BSA, the Software and Information Industry Association (SIIA) is a trade-group advocating and defending the rights and interests of big software companies. Notable members include, amongst many others, Apple, Facebook, Goldman Sachs and even Credit Suisse. The SIIA has the stated goal of promoting, protecting and informing the software and digital industry. Also similar to the BSA, it has an anti-piracy reward program offering up to 1'000'000\$. \parencite{SIIAreward}

Critics accuse the SIIA to use the complex and confusing IP laws to extort money from small- and medium sized businesses in what is essentially a shakedown. \parencite{Melymuka2006}

\subsection{EFF}
\label{sec:EFF}
The Electronic Frontier Foundation (EFF) is a non-profit organisation and was founded in 1990 as a response to a series of raids performed by the United States Secret Service. These raids were done to track an illegally distributed manual for the emergency call system. In the course of these raids the business of a Mr. Steve Jackson was nearly driven to ruin, as it had its computer equipment confiscated, searched and tampered with. Mr. Jackson turned to civil rights groups to file a lawsuit against the Secret Service, but none of the contacted ones knew enough about the issue. Hearing about the raid and subsequent damage to Mr. Jackson's business, the three founding fathers (John Perry Barlow, John Gilmore and Mitch Kapor) announced the creation of the EFF and their intent to represent Mr. Jackson in court.
The EFF continued to file lawsuits and represent individuals in the upcoming decades. Their focus concern privacy rights, anti censorship, free speech and copyright issues. \parencite{EFFhistory} \parencite{Schultz}

\subsection{CC}
\label{sec:CC}
The Creative Commons (CC) is a non-profit organisation founded by Lawrence Lessig (a Stanford law professor) with the help of the Center for the Public Domain. Upon founding, CC released a set of licences; using these, rights holders can release works to the public with various degrees of freedoms and opportunities for derivative works or permission to use this work.
In 2005 the CC launched the open science initiative with the goal to provide the chance to collaborate and contribute. In this program, research data, lab notes and other research material are freely available and can be reused, redistributed and reproduced.
The present CC has grown to be an international network of partner organisations with like-minded goals and motivations. \parencite{Plotkin2002} \parencite{CC2016}
Critics accuse the CC to be too industry friendly and not doing enough for the common creativity, as the organisation states. \parencite{Berry2005}

\chapter{Current situation}
\label{ch:CurrSit}

\section{Business Models}
\label{sec:BusMods}

\subsection{Open Source Software}
\label{ssec:OSS}
According to Wikipedia, Open Source Software (furthermore abbreviated as OSS) is " computer software with its source code made available with a license in which the copyright holder provides the rights to study, change, and distribute the software to anyone and for any purpose."\parencite{WikiOSS}. One of the founding pillars and most cited sources regarding the motivation for OSS is the GNU Manifesto of Richard Stallmann, written in 1985 wherein he states among other things that "I consider that the Golden Rule requires that if I like a program I must share it with other people who like it. [...] I refuse to break solidarity with other users [...]." (citation GNU Manifesto https://www.gnu.org/gnu/manifesto.en.html).

The idea was later formalized into the four freedoms that make software free software as postulated by the Free Software Foundation: "When you speak about Free Software, you speak about freedom. And more precisely, about the four freedoms to use, study, share and improve the software." (citation needed https://fsfe.org/freesoftware/basics/4freedoms.en.html).

This idea is in stark contrast to proprietary (closed source) software. One might conclude that OSS thus is a hobby, not a business, as you can't sell the same software anyone can download and compile. This, however, is not true and was already anticipated at the time of the GNU manifesto, which states: "If people would rather pay for GNU plus service than get GNU free without service, a company to provide just service to people who have obtained GNU free ought to be profitable." (citation GNU Manifesto https://www.gnu.org/gnu/manifesto.en.html). The business model around OSS can thus be characterized as selling services to customers. There are several services that can be provided for a charge, e.g.:
\begin{itemize}
	\item Support
	\item Enhancements
	\item Customisations
	\item Education 
\end{itemize}

The question is how one can credibly offer good services and be competitive when everyone else could do the same? It comes down to reputation. If someone has contributed to the programming of a software, this person (legal or natural) gets recognised for that effort (see also https://en.wikipedia.org/wiki/Open-source\_software\_movement). This recognition of the effort and the knowledge of a system can be used to bring leverage to the marketing efforts for selling services around the software. Thus, it is not surprising that big corporations have stepped in to contribute to OSS as the example of Linux, a free operating system, clearly shows. In their annual announcement of the Linux Development Report 2015, the Linux Foundation mentions "The Top 10 organizations sponsoring Linux kernel development since the last report include Intel, Red Hat, Linaro, Samsung, IBM, SUSE, Texas Instruments, Vision Engraving Systems, Google and Renesas." (citation https://www.linuxfoundation.org/news-media/announcements/2015/02/linux-foundation-releases-linux-development-report). While Suse and Redhat are primarily Linux distributors, others use Linux in their products and services. The example of Google, IBM and Samsung illustrate that the interest in OSS not only comes from small and medium-sized enterprises, but also big, global corporations.

The programmers themselves are often paid to work on the OSS, as the same report anouncement also states: "The number of paid developers is on the rise, as companies aggressively recruit top Linux talent. More than 80 percent of kernel development is done by developers who are being paid for their work. Volunteer developers tend not to stay that way for long."

\subsection{Software Licensing}
\label{ssec:SLic}
Software is often made available using licenses. According to Merriam-Webster, the general meaning of a license is a "permission to act", respectively a "freedom of action"\parencite{WebstLic}. One of the subordinate meanings, again according to Merriam-Webster is "a grant by the holder of a copyright or patent to another of any of the rights embodied in the copyright or patent short of an assignment of all rights". 

In the software business, the meaning is that you do not buy a copy of the software, but the rights to use the copy of the software, or as defined in Wikipedia: "A typical software license grants the licensee, typically an end-user, permission to use one or more copies of software in ways where such a use would otherwise potentially constitute copyright infringement of the software owner's exclusive rights under copyright law."\parencite{WikiSoftLic}. To be entitled to use such proprietary software, the users typically have to agree to compensate the copyright holder for obtaining a copy, which seems quite reasonable. 

Furthermore, the user is most often required to agree to software license agreements, stating further limitations to which the license of the software is subject to. This mechanism is summarized in Wikipedia as following: "Proprietary software licenses often proclaim to give software publishers more control over the way their software is used by keeping ownership of each copy of software with the software publisher. By doing so, Section 117\footnote{17 US Code § 117 defines rights that an owner of a rightful copy has, such as transferring or incidental copying (into RAM during use). This can be understood that an owner of a copy can use that copy without regard to having a license.} does not apply to the end-user and the software publisher may then compel the end-user to accept all of the terms of the license agreement, many of which may be more restrictive than copyright law alone."\parencite{WikiSoftLic} These terms might e.g. limit the duration of the license to one year, prolonged only in case of further payment. Microsoft e.g. grants the rights to use Office 365 only for the duration of a year, after which a repeated payment is required\parencite{Off365}.

\section{Mentality concerning IP rights}
\label{sec:IPMent}

As mentioned in the section \ref{sec:Patents}, the use of patents and copyright was to 'promote the Progress of science and useful arts'. This was adapted from the british system forged in 1710; under this system only books, maps, and charts were protected by copyright. \parencite[3]{Cummings2010}

These limitations were also applied to works produced in America. Some of the works not protected were granted protection afterwards (such as dramas and sheet music). Interestingly enough, even these were not eligible for copyright protection, if they were found to be of illicit nature.  Exactly this happened to the work \textit{Three Weeks} by Elinor Glynn, which was deemed so \textquotedblleft grossly immoral that copyright protection cannot exist\textquotedblright\ as is described in \cite{Fox1945}.
We can conclude that piracy was, if not necessarily as widespread as it is today, prevalent enough that voices for judicial protection never ceased to exist.

The same was true for the creators of music in the 1900s, as sheet music was not deemed \textquoteleft useful arts\textquoteright\ and therefore not worthy of protection. Even after sheet music was being protected by the 1909 copyright act, which introduced the ingenious construct of 'compulsory licensing' (in which the creator can choose who will produce the first recording of his music, but every other recorder just has to pay a flat fee), the recordings of said music were not itself protected. The justification of this disparity was that these recordings were merely parts of a machine and thus not a creation protected by copyright.

This led to a proliferation of copied LPs. Especially bootlegging rare Jazz albums was common in the 1930, and even tolerated by the major labels as one executive told: \textquotedblleft Ah, it wasn't worth the trouble to put out that moldy stuff [...] It never sold anyway\textquotedblright\ \parencite{Cummings2017}
In the 1960 the flourishing of pirated music on magnetic tape and vinyls prompted the music industry to push for another try at protecting their products. This time around their narrative was not some ideal goal of authorship, but their own bottom line and how piracy threatened the American industry and jobs. This push was more successful than the preceding ones and lead to records being protected under American law.

In the same time, the efflorescence of radio and television led to the development of, what \cite{Lin2013} calls, the mentality of free. The customer is receiving content for \textquotedblleft free\textquotedblright\ from service providers, which in turn pay the licensing fee and use advertising to pay these and profit from their service. This business model was broadly adopted in the age of internet. The problem being that the customer had to listen or view advertisements on radio or television; banner ads, however could be more easily ignored. During this time Peer-to-peer filesharing systems like Napster started to spring up, propelled by the ever increasing speeds of domestic internet connections. And quoting from the excellent paper from \cite{Lunceford2008}: \textquotedblleft The logic of the consumer dictates that if a product is freely available on the market [through service providers], it is probably  legal.\textquotedblright

In the 2000s the RIAA was beginning to crack down on these so called file sharers. But in their scorched earth approach which not just targeted big copyright infringers (or so-called pirates), but also twelve year old kids, the music industry garnered the ire of the public. The use of the term piracy is controversial itself, due to the very different nature of real piracy (which is often lethal high-seas robbery) to the act of file sharing. The public did not buy into their narrative that the music industry was the victim in this and the copyright deadlock which copyright owners now possessed (life of author plus 70 years) made it seem like just another cash grab by a powerful conglomerate. Another thing complicating the matter is people identifying themselves with the music, so it not longer belongs just to the creator and owner of copyright in the mind of the public. \parencite{Lunceford2008}

\section{Globalisation \& increased complexity of rights management}
\label{sec:GlobalRightsMgmt}
The global development of industries over the past few decades had quite a significant impact related to patents. Companies from all over the world sometimes hold patents relevant to whole industry branches. Furthermore, the number of licensed patents used in modern products and services has take on a new dimension. One example for this is the LTE \footcite{Long Term Evolution} standard for mobile communications: Differing opinions and a multitude of affected patents belonging to different technology companies led to several law suits between these parties and a delay of years until the adaption of this standard.

\subsection{Patent magnitude}
As technology gets more complex, more companies are involved in development of individual elements that make up standards, services and products of our modern age. And since new technologies are often derived from established ones, it comes as no surprise, that some modern products are related to astounding numbers of patents. As an example, we can take a look at the 3GPP mobile communications standard. It is estimated that there are over 8000 patents considered essential for this industry standard, where approximately 90\% of those are split between 12 companies \parencite{Wiki3G}. It is clear that with such large pools of patents relevant to industry standards, proper licensing is a complex matter.

The follow-up standard LTE is no exception either. While it is not easy to find exact numbers on the size of the patent pool related to the standard, its effects were rather visible over the past few years. A number of lawsuits have been filed over standard relevant patents: Huawei versus ZTE \parencite{HuaweiVsZTE}, Apple versus Ericsson \parencite{AppleVsEricsson}, Huawei versus Samsung \parencite{HuaweiVsSamsung} and Huawei versus T-Mobile \parencite{HuaweiVsTmobile}, just to name a few.

\subsection{Cross-licensing}
To limit the risk of patent infringement lawsuits, many companies enter cross-licensing agreements. Parties involved in a cross-licensing agreement grant each other a license to use patents related to the subject matter of the agreement. Very often each party owns patents related to different aspects of a given product. Thus a cross-license maintains their freedom to bring new products to market \parencite{CrossLicense}.

A main limitation of cross licensing is that it is ineffective against non-practicing entities (NPE). Their primary business model is to acquire patents, and exploit these for monetary royalties. Thus, they have no benefit in obtaining the rights to use other companies' patents. Today NPEs are often referred to as patent trolls \parencite{PatentTroll}. 

\subsection{Reasonable and non-discriminatory licensing}
To manage the sheer magnitude of patents involved in modern standards, a licensing scheme based on reasonable, non-discriminatory terms (sometimes also fair, reasonable and non-discriminatory terms) has been adopted by standard setting organizations and accredited by courts.

This refers to a voluntary licensing commitment often requested by standards organisations from holders of patents that might become essential to a technical standard. Fair relates to the underlying license terms, usually based on anti-trust laws to prevent anti-competitive demands of the licensee. Reasonable refers to the height of license rates. Non-discriminatory relates to both the terms and rates. As the name suggests, it means that a licensor treats different licensees under similar terms and rates \parencite{WikiRAND}. 

Terms for RAND licensing are often based on a decision of the Federal Court of Justice of Germany from 2009. The court ruled that a defendant accused of patent infringement who was not able to obtain a license from the licensee may defend himself by invoking an abuse of a dominant market position \parencite{OrangeBook}.
 
Another important case related to RAND is Microsoft versus Motorola. In 2012, the United States Court of Appeals for the Ninth Circuit ruled that a patent holder's agreement to a policy like RAND creates a legally binding contract \parencite{MicrosoftVsMotorola}.

\chapter{Discourse}
\label{ch:Disc}

\section{Copyright in the context of the arts}
\label{sec:CopyArts}

\subsection{Pro IP}
In \cite{Towse1999} it's highlighted that the production of a work, conceiving ideas, ordering them and structuring them to a finished work, takes time and energy. To incentivise this investment of energy is the main goal of copyright. In granting the creator of a work the right to control and restrict the acts of copying (for example copies, but also performances and broadcasts of said work), it ensures that future works can be produced.

It also highlights the costs necessitated for the production of a sound recording. This initial cost is not needed by a reproducer of the same work. Copyright thus also protects the publisher, which usually fronts the cost as an investment, from being undercut by competitors which do not have to carry that initial cost of producing a record. The transfer of copyright from creator to publisher, can be organised in a multitude of schemes. The most prominent ones are royalty (in which the creator gets a share from the sale of the work) and 'buy-out' arrangements (in which the creator gets one payment for all his rights), and Towse does not fail to highlight the skewed power dynamic between these two parties (publishers have more possibility to distribute risks through a bigger portfolio of individual creators, compared to creators who only have their work and own human capital).

Still, that works continue to be produced there needs to be a set of rules who reward the creators and first enterpreneurs for taking the risk and investing resources.

\subsection{Against IP}
There are several possible problems with copyrights in the context of the fine arts that come to the mind of the authors of this thesis. 

Let us first look into the area of music.
\begin{itemize}
	\item There are 12 tones in the chromatic scale, and normal scales consist of 8 tones. If we consider that these have been used for hundreds of years, there is a limitation of the number of possible arrangements in a certain cycle. This means that - whatever a composer does - there is a high probability that parts of the composition exist in parts of other compositions. This does not mean that no original compositions are possible anymore, but that it is quite certain that there are overlaps with other musical pieces. If one considers that the teachers in the craft use similiar teaching techniques (as e.g. using variations of the scales for guitar solos) as well as the fact that we hear music without being consciously aware of it rather often, e.g. in shopping malls or passing someone using a radio, the possibility of unintentional inspiration is high. 
	\item In the current age, music is often not the creation of an inspired artist but of composers creating goods based on market research, e.g. the current music taste and characterizations of idols for pre-teen and teenage girls, which does lead to floods of one-hit wonders and good singers that get burnt out for the financial gain of industry giants. As such one-hit wonders generate big revenue streams for the publishers that have also controlled the "compositions" (and I use that word loosely) and only have to pay for the actual performance and minor wages for the singers and musicians, the market gets saturated by low quality music leaving less space for really talented artists and making it hard for them to get acceptable contracts, as the risk for the publishers might be higher and the part of the revenue going to the artist is (rightfully) higher, hurting the industry's bottom line. This, however, leads to good artists giving up or not being able to generate an income by touring, as their music is not known enough, basically being drowned by low-quality noise. This might function short-term, but makes the cultural landscape that much poorer for it. Were it not for copyright allowing huge short-term gains, market researched one hit wonders would not work because the demand for viewing concerts of such market-oriented music would cease quite quickly, thus not generating enough money to make it worthwile.
\end{itemize}
Another area is literature. There are many known genres like fantasy, western, science-fiction, history dramas and so on, setting the stage for the plot. And there are basic truths that are known since the advent of oral story tellings that come to mind.
\begin{itemize}
	\item A story traditionally has one or multiple leading characters that are followed through the plot.
	\item The characters have relations to other persons.
	\item Most often, there are opponents and antiheroes.
	\item The leading characters use their available means to deal with the cards they are given in live (e.g. magic or swordplay in fantasy novels, colts and guns in westerns, special animals like dragons, gryffins, horses etc...)
\end{itemize} 
These are basic rules in storytelling and have been for centuries. Additionally, hundreds of thousands of books have already been written. Should an inspired teenager no longer be capable of writing about his fantasies like learning magic, just because there have been a series of very successful books about similiar settings like in Harry Potter? After all, the dream is similiar to another plot and some parts of the settings have also been used in such a context. Can heroes no longer ride on dragons in fantasy novels because the books of Eragon have been successful? Is such a world where similiar plots can't be written and distributed not that much poorer for it?

While the aforementioned problems are about the creation of works, other problems exist concerning already existing works. The terms of copyrights are often defined as during the life of the creator plus 50 to 70 years. Let us think about a book that had not been very successfull in the sixties that would fit into the current climate (political and cultural). While the production of such a book might have stopped during first edition, leaving only few actual copies, one could not legally copy such a book for oneself even if the author is dead and the publisher is not interested in printing it. Over time, the last copies of the book might disappear leaving no trace of a possibly great work that had not been recognized during the time of its creation, leaving the world a poorer place.

As a last argument, the author wants to mention copyright trolls like Prenda Law or Malibu Media, that are suspected (and in the case of Prenda Law even convicted) of seeding porn to bit torrent themselves to be able to exact settlement payments in such a way that even people not having downloaded that content would rather pay than having to face a costly lawsuit, alleging them to be porn users. While the author has selected the most extreme case as illustration, there are numerous other ways that copyright trolls exact payments in other areas of arts, very often without a real basis, because the defendants cannot hope to bring up the money to defend themselves against a corporation with deep pockets.

\subsection{Addressing Against IP's points}

When we are talking about 12 tones, then we are very western centric. Oriental and Asian music for example uses other tones. But even if we narrow our field we would not run out of melodies for quite some time \parencite{Stevens2012}, or as Prof. Richard Cohn said: "the number of combinations is so large as to be infinite for any practical purpose" \parencite{Niller2015}.
I would concede that the problem with pop music lies in the self-devourment of its most successful tracks, which might lead to a notion that all songs sound the same.
Still, I can't find an argument in this paragraph which would speak for the abolition of copyright. And I like to reiterate, with the possibilities of self-publishing which the internet provides small artists need to be protected, and I think copyright fits this job nicely.

One can argue that every story is just a flavour of "The Hero with a Thousand Faces" by Joseph Campbell, but this would disenfranchise the work done by him, and also misses the mark of a good story. Ask every reader of a book on what they are invested in, and I can guarantee you it will always be "this character". Be it that the one in question is most identifiable with or has virtues one strives after, characters are the spice that make a story worthwhile. And the creation of these should be protected as well.

At last, I'd like to agree that the big D company will never let the mouse die. And that the terms of copyright (life of the author plus seventy years) are frankly absurd. Same story with the aforementioned copyright trolls. But, I would say the problem lies with the abuser and not the law itself (or at least with the core idea of the copyright law). I'm no scholar of the law, but I would surmise that root of the problem lies in the American system of government itself, and will not be easy to do solve.
Still, especially small authors and artists need the protection from copycats and plagiarists, but I can agree that the copyright law in it's current form does a bad job at that.

\subsection{Adressing Pro IP's points}
We can accept that  creating a real work of art is a process that takes time and energy and should be rewarded. This is exactly where the problem lies, however. Real artists doing this work have a hard time competing in an environment, where cheap remixes and computer produced tunes are produced thousandfold by major labels. These tunes presented by singers that are never heard of afterwards strangle and saturate the market with so-called music that noone ever listens to anymore after the hype has faded away. One example for this kind of "music" productions are works presented by winners of casting shows and boygroups that seldom survive half a decade.

While the costs of music production used to be huge, newer technologies and distribution models have decreased these costs to a level that could be affordable for true artists, were it not for a saturation of the market by low-quality one-hit wonders. Those true artists then can't reach the audience enough to make touring and merchandise profitable.

\section{Copyright in the context of the creation of software}
\label{sec:CopySoft}

\subsection{Pro IP}
An average software user is not a developer and in many markets does not have technical expertise related to the workings of software they use. For those users a software product is just a tool they integrate into their work-flow. As a business end user their time and productivity has a direct cost. Therefore many users hold training, documentation and support in high regards. Proprietary software vendors have a high incentive to offer training and support. A seizable potion of the cost to acquire and license their products is for support. 

Larger corporations often need to dedicate a substantial budget to employee training. Using software that is a de-facto standard, for example Microsoft Office, can save the corporation enough money in training cost to offset the cost related to the software. Simply because many of their newly recruited employees already have some level of expertise with the software due to its market penetration.

Last but not least security is a big concern in the software world. Proprietary software is developed in a concentrated team. Allowing to budget and implement a security process with periodic and verifiable peer-reviews and security audits of the source code. 

\subsection{Against IP}
In \ref{ssec:SLic}, the fact that terms limiting the freedom of the user can be required to get a license for a product, with the example of Office 365 that requires a yearly payment. In fact, Office 365 ceases working after a certain period without subscription. Imagine a person having lost his job and not being able to pay that subscription, having his outdated CV only in electronic form, as a Word document. He would, in the time of his greatest need, be able to update his CV properly, and being not able to pay the internet provider, this person might not be able to download a free text processing software, even if he knew about one. 

The author of this chapter is a firm believer in open source software. Innovation can no longer exist in a vacuum, because so many things already exist. And innovation is important for the advancement of human culture. The digital age would not have been realized the way it has been were it not for Google. Google on the other hand would not have succeeded were it not for OSS. The costs of proprietary operating systems for server farms the size of Google's would have made the rise of the search engine impossible. Another reason that OSS is to be preferred to closed source software (called CSS from here on) as described in \ref{ssec:SLic} is the data that is created and managed by the software. If a software publisher goes out of business, the software will no longer be updated or maintained. Within two or three operating system lifecycles, it is no longer possible to run that software at all, thus making all the data created worthless, because it is no longer readable. While standards can weaken this effect, not every software adheres to a standard, so the binary file formats are useless. Imagine using a comfortable database application for managing your family photos, and you lose the photos, just because the software is no longer available to read the archive. Such a scenario is realistic, but should not be.

Another problem is the applicability and scope of copyrights. Are APIs copyrightable? If so (and there are lawsuits claiming this is the case), should it no longer be possible for someone to create a programming language in an object-oriented manner, because constructors and other programming constructs would logically be implemented in a certain way? 

To the author of this chapter, the cost of CSS to the advancement of the sciences is too big and the benefit too small to accept copyrights in the area of software creation. As mentionned in \ref{ssec:OSS}, it is possible to make money (make lots of money) when working with open source software.

\subsection{Addressing Against IP's points}
One of the key advantages of open-source software is the ability to modify the code. But many users don't have the need, let alone the programming skills to so. Even if modifying the code is desired doing so always incurs costs, at the very least in work hours. A developer needs to learn the project structure, learn naming and coding conventions, understand where the extensibility points are and often have to learn about additional open-source libraries. All this before actually writing the code for the desired modification. 

Furthermore these changes then need to be maintained. With every new release of the underlying software changes made need to be tested and often modified. Otherwise the user runs the risk of relying on outdated, potentially buggy or insecure software just because maintaining customizations is not profitable enough. This adds risks to the software used and potentially projects depending on it.
 
A major part of the incentive for a developer to work on a open-source project is to gain recognition as the author. For very small, specialized markets this incentive diminishes drastically. 

Of course there are a lot of open source projects with big communities of developers and users, as the afore mentioned free operating systems. But there are areas where the market is so small that it is unlikely that such a community will form, like the broadcast industry. Its users are not developers, so they cannot help write the software they need. Furthermore, those users need easy to learn and easy to use software that helps them achieve the productivity they need. Training and ease of use are often lacking with open source projects, especially small ones.

Many open-source projects do not have a budget they can use on usability and documentation, consequently falling short on these. Also user support often relies on communities like forums. An eventual reply to a forum post cannot be compared to a guaranteed reply to a phone call. If your productivity depends on a quick resolution of issues, quick and guaranteed (through service level agreements) answers cannot easily be replaced by a community of volunteers. 

Of course there exist support suppliers for open-source software. Naturally their service is not free and since these suppliers cannot subsidize the support costs with product sales, since the product is distributed free of charge, these fees tend to be quite high.

\subsection{Adressing Pro IP's points}
It is true that the average software user does not have much technical expertise related to the inner workings of software. However, the incentive to offer good training and support is even stronger in the OSS market, as this is the primary mode of generating income. Additionally, there are many corporations competing in offering exactly that, e.g. basic Linux training by IBM, RedHat, Linux Foundation and others, often following the same structure from LPI\footnote{LPIC is short for Linux Professional institute}, but differing in location and course mode.

While it is unfortunately true that the office software of Microsoft is a de-facto standard, it only seems to be beneficial to the companies at large. There is an advantage of having software that many employees already know, but the disadvantage is that other, competing products do not have a chance to innovate in that market, as entering or either disrupting this market is prohibitively expensive.

The argument about security, however, falls miles short. The only possible security advantage would be security by obscurity, which does not work if we look at the current state of the IT environment. While hackers can exploit vulnerabilities they have found or bought, the security crowd can only react after a vulnerability is published or exploited, leading to situations where a vulnerability that was known but hidden by the American Intelligence community can be exploited to create massive damage and e.g. shut down hospitals, just as the WannaCry suite of malware has done.

\section{Patents}
\label{sec:DiscPatents}

\subsection{Pro IP}
\label{ssec:ProIPPatents}
In their 2010 study \citeauthor{Artz2010} consolidate findings, that patents are positively related to product announcements, and in turn also to innovation. This does not mean, that patents necessarily drive innovation and invention, but that they are used to protect products derived from the inventions and help companies recuperate spendings invested in the development process. 

It has to be said, that patents itself are relatively unimportant to innovation itself. Outside the pharmaceutical and chemical fields (a finding first observed by \citeauthor{Mansfield1986}), they are usually used strategically to block competitors innovation and delay their ability to imitate the product. 

As in most fields (apart from the previously mentioned) one can "invent around" a patent relatively cheap, and the cost to prove that a specific patent has been infringed is relatively high. Nonetheless, are patents an important economic instrument to ensure a return on investment of committed resources.

\subsection{Against IP}
Patents should no longer have a place in our globalized and digitized world. The cost seems too high. In this globalized world, innovation happens at a far faster pace than when patents have been introduced. Patents no longer offer just an edge to inventors compared to their competitors, they are far too often used to eliminate the competitor and in fact discourage persons when it comes to the question if they should start a company based on an idea.

Furthermore, there's nothing new anymore, only new combinations of existing techniques. The electric car is just a combination of battery technics, electronic motors and computers with concepts of existing cars. Drones are merely more than a combination of wireless communication, well known aerodynamics, and so on. Smartphones are merely a mixture between computers and telephones, and so on. Unfortunately, the patent system does not really address that. While there is a notion that one can't patent combinations that would come to mind if persons of the same arts of science look at the same problem, this obstacle is rarely even considered when patent officers grant patents.

The society should ask itself if it wants to cater to the needs of greedy corporations and believe their propaganda, or if not the need to solve problems would lead to the discoveries of things without the financial incentives that patents offer. Society should ask itself, where its priorities should be: in advancing to a better future or in deepening deep pockets.

\subsection{Addressing Against IP's points}
While I agree that strategic patent filing (as described in the section \ref{ssec:ProIPPatents}) and the methods of non-practicing entities is probably not in the sense how the drafters of the patent law have envisioned it, the notion of doing away with patents altogether is flawed. 

Developing new medicines, for example, is a long and cost intensive process, and the new compound can be analysed relatively easy (or at least for a fraction of the cost it took to develop the compound). And usually the successes have to pay for the failed projects, which drives the costs up \parencite{Herper2012}. Were it not for patent protection, no company would invest the money to develop a new medicine or chemical.

If we could divorce money from research, then patents might as well be non existent, but this is not how the world works today and more a philosophical question than a real argument.

Responding to the "nothing new anymore, only new combinations of existing techniques": As Bernard of Chartres has said, we are just dwarfs perched on the shoulders of giants. Of course we are using technologies and techniques developed by our forebears, but to repudiate all new inventions as just variation of the old is foolish and naïve. Is a Smartphone just another Natel, and the Natel itself just a portable phone?

\subsection{Adressing Pro IP's points}
The Pro IP chapter of patents mentions the possibility to recuperate spendings invested in the development process. This might apply, although development nowadays is mostly combining existing technologies to new products and protocols. However, even the Pro IP chapter mentions the use of patents to block innovation from competitors, which decreases the progress of technology as a whole, a cost that cannot be offset by the recuperation of compiling existing technologies to something new.

\chapter{Conclusion}
\label{ch:Concl}
% Paragraph on the arts
% Nati writes this

% Paragraph on the software
% Reto writes this
When it comes to copyright in the realm of software it is hard to take a clear stance for or against closed source programs. The potential benefits of open source software are apparent. The linux operating system, the apache webserver and a number of protocols and supporting software that form the backbone of todays world wide web are clear examples, that high quality software can be developed and maintained as open source and even become industry standard. Also companies like Google and Facebook have proven, that it can be beneficial to all -- even themselves -- to make some of their internal work available for free. But those examples can also serve to illustrate the downside of open source. When you look at smaller, more obscure projects without such a large user base, the picture becomes much more fuzzy. Small projects can be of high quality too of course, especially when a few companies see enough value in them to invest resources of their own, but that is not always the case. This becomes even more an issue for software not targeted at IT professionals or power users with high technical expertise. This can lead to nieche markets where open source software simply does not meet user requirements or does not emerge at all. But even in such markets closed source software may thrive.

% Paragraph on the patents
% Pascal writes this

\newpage

\printbibliography

\end{document}          
