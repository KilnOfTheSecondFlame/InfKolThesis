\documentclass[a4paper]{report}
% Use swiss german letters
\usepackage[utf8]{inputenc}
% Language: english
\usepackage[english]{babel}
% Blindtext package
\usepackage{blindtext}
% Fancy Figures
\usepackage{graphicx}
% Display the Bibliography in the TOC
\usepackage{tocbibind}
% Better lists
\usepackage{enumitem}
% Use biblatex
\usepackage[style=apa,backend=biber,citestyle=authoryear]{biblatex} 
% Define the bibliography file
\addbibresource{bibliography.bib}
% To let LaTeX handle "
\usepackage[autostyle, english = british]{csquotes}
\DeclareLanguageMapping{english}{english-apa}
% To have text wrap around pictures
\usepackage{wrapfig}

% Titlepage
\newcommand*{\titleAP}{\begingroup % Create the command for including the title page in the document
	\centering
	\vspace*{\baselineskip} % Whitespace at the top of the page
	
	{\Large Pascal Baumann} and {\Large Nathanael Gogniat} and {\Large Reto Scheidegger}\\[0.167\textheight] % Author name
	
	{\Huge\bfseries Intellectual property in the digital age}\\[\baselineskip]
	
	{\Large \textit{Thesis paper InfKol FS2017}}\\
	\today
	
	\vspace*{3\baselineskip} % Whitespace at the bottom of the page
	\endgroup}

\graphicspath{{./img/}}

\begin{document}

\titleAP

\begin{abstract}
	\blindtext
\end{abstract}
\newpage

\tableofcontents

\newpage

\chapter{Introduction}
% TODO remove - Labels are used for in book referencing - ie. if you want to reference this chapter you would use \ref{ch:Intro}
\label{ch:Intro}

\section{Copyright}
\label{sec:Copyright}

\section{Patents}
\label{sec:Patents}

\section{Trademark}
\label{sec:Trademark}

\section{Organisations}
\label{sec:Orgas}

\section{Digital Age}
\label{sec:Digital age}
The term "Digital Age" is a description of the current period. The Cambridge Dictionary describes it as "the present time, when most information is in a digital form, especially when compared to the time when computers were not used" (citation needed for http://dictionary.cambridge.org/de/worterbuch/englisch/digital-age visited on 02.03.2017).
 
The definition found in Wikipedia goes a bit further by incorporating the business impact and the article starts as following: "The Information Age (also known as the Computer Age, Digital Age, or New Media Age) is a period in human history characterized by the shift from traditional industry that the Industrial Revolution brought through industrialization, to an economy based on information computerization" (citation needed for https://en.wikipedia.org/wiki/Information\_Age visited on 02.03.2017).
   
\subsection{Influence of the Digital Age on IP}
The digital availability and the instant accessibility of the information (anytime from anywhere) is in stark contrast to the availability of information in earlier ages. This becomes obvious when looking at the example of books as means of information distribution in Europe.

During the period of antiquity, information was written manually on tablets, scrolls and similiar media. Thus, saving information was a time consuming and burdensome task.  Later on, books with pages of parchment became the norm. These books were copied manually by monks or other scholars. As a consequence, these items were rare and expensive. Later on, the single pages of a book were distributed to different copyists to increase the speed of copying. Although this parallelisation of copying made copying a book faster, the effort used was only distributed and not minimised; thus, books remained rare and expensive. This began to change after Johannes Gutenberg invented the printing press with movable type around 1450. This development sped up the process of copying books, because the same page could be printed multiple times quickly after the set up of the print plate was completed. However, copying still required the presence of a physical specimen to copy from. With an increasing number of now affordable books, the knowledge contained could slowly spread further, but was still hindered by long, slow and dangerous transports. While the advancement in printing and transportation sped up the process of copying and transportation later on, instant access to information was still not available because the information was bound to a certain geographical location. Before the rise of the internet, people still had to go to libraries for research purposes, oftentime being required to order specimens of the books from other libraries to the closest one - or order a personal copy of a book.

With the establishment of the internet as a commodity, available not only from home and the office but from nearly everywhere via mobile devices, the point of origin of a published text no longer matters for the availability everywhere. The distribution of the information no longer happens in concentric regions around a physical specimen but can become available instantly everywhere at the same time. Getting a copy of a book no longer requires burdensome printing and transport but can be done easily doing some mouse clicks. While this can be beneficial to the human race as a whole, the originator of the books (now in digital form) is no longer sure that he will be compensated, as a book may be distributed using file sharing for free as soon as one copy has been sold.



\chapter{Conclusion}

\newpage

% TODO remove - The bibliography is only printed if there are quotations in the text
\printbibliography

\end{document}          
