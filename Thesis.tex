\documentclass[a4paper]{report}
% Use swiss german letters
\usepackage[utf8]{inputenc}
% Language: english
\usepackage[english]{babel}
% Blindtext package
\usepackage{blindtext}
% Fancy Figures
\usepackage{graphicx}
% Display the Bibliography in the TOC
\usepackage{tocbibind}
% Better lists
\usepackage{enumitem}
% Use biblatex
\usepackage[style=apa,backend=biber,citestyle=authoryear]{biblatex} 
% Define the bibliography file
\addbibresource{bibliography.bib}
% To let LaTeX handle "
\usepackage[autostyle, english = british]{csquotes}
\DeclareLanguageMapping{english}{english-apa}
% To have text wrap around pictures
\usepackage{wrapfig}

% Titlepage
\newcommand*{\titleAP}{\begingroup % Create the command for including the title page in the document
	\centering
	\vspace*{\baselineskip} % Whitespace at the top of the page
	
	{\Large Pascal Baumann} and {\Large Nathanael Gogniat} and {\Large Reto Scheidegger}\\[0.167\textheight] % Author name
	
	{\Huge\bfseries Intellectual property in the digital age}\\[\baselineskip]
	
	{\Large \textit{Thesis paper InfKol FS2017}}\\
	\today
	
	\vspace*{3\baselineskip} % Whitespace at the bottom of the page
	\endgroup}

\graphicspath{{./img/}}

\begin{document}

\titleAP

\begin{abstract}
	\blindtext
\end{abstract}
\newpage

\tableofcontents

\newpage

\chapter{Introduction}
% TODO remove - Labels are used for in book referencing - ie. if you want to reference this chapter you would use \ref{ch:Intro}
\label{ch:Intro}

\section{Introduction}
\label{sec:Intro}

\section{Digital Age}
\label{sec:Digital age}
The term "Digital Age" is a description of the current period. The Cambridge Dictionary describes it as "the present time, when most information is in a digital form, especially when compared to the time when computers were not used". \parencite{CambridgeUniversityPress2014}

The definition found in Wikipedia goes a bit further by incorporating the business impact and the article starts as following: "The Information Age (also known as the Computer Age, Digital Age, or New Media Age) is a period in human history characterized by the shift from traditional industry that the Industrial Revolution brought through industrialization, to an economy based on information computerization \parencite{WikiInfoAge2017}

\subsection{Influence of the Digital Age on IP}
The digital availability and the instant accessibility of the information (any time from anywhere) is in stark contrast to the availability of information in earlier ages. This becomes obvious when looking at the example of books as means of information distribution in Europe.

During the period of antiquity, information was manually written on tablets, scrolls and similar media. Thus, saving information was a time consuming and burdensome task.  Later on, books with pages of parchment became the norm. These books were manually copied by monks or other scholars. As a consequence, these items were rare and expensive. Later on, the single pages of a book were distributed to different copyists to increase the speed of copying. Although this parallelisation of copying made the process of copying a book faster, the effort used was only distributed and not minimised; thus, books remained rare and expensive. This began to change, after Johannes Gutenberg invented the printing press with movable type around 1450. This development sped up the process of copying books. Because, after the set up of the print plate was completed, the same page could be printed multiple times in a timely manner. However, copying still required the presence of a physical specimen to copy from. With an increasing number of affordable books, the knowledge contained could be spread further. But this process of distribution was still hindered by long, slow and dangerous transports. While the advancement in printing and transportation sped up the process of copying and distribution later on; instant access to information was still not available, because the information was bound to a certain geographical location. Before the rise of the internet, people still had to go to libraries for research purposes, often-time being required to order specimens of the books from other libraries - or order a personal copy of a book.

With the establishment of the internet as a commodity -- available not only from home and the office, but from nearly everywhere via mobile devices -- the point of origin of a published text no longer matters for the global availability. The distribution of information no longer happens in concentric regions around a physical specimen, but happens instantly, everywhere, and at the same time. Getting a copy of a book no longer requires burdensome printing and transportation, but can be done easily by doing some mouse clicks. While this can be beneficial to the human race as a whole, the originator of the books (now in digital form) is no longer sure that he will be compensated, as a book might be distributed freely using file sharing as soon as one copy has been sold.

\section{Copyright}
\label{sec:Copyright}

\subsection{Definition and scope of copyright}
\label{sec:CopyDef}
Copyright is a form of intellectual property right generally associated with works of artistic value. It is intended to fight plagiarism. The legal term copyright describes the rights a creator holds over their literary and artistic work. Such works include books, paintings, movies, computer programs, databases, advertisements and maps. Copyright usually only applies to a practical expression and not ideas, procedures or mode of operation. \parencite{CopyGov}

Copyright includes economical and moral rights. The economical rights allow the rights holder to derive financial reward from the use of protected works. The holder of economic rights can authorize or prohibit reproduction, broadcast, public performance, recording, translation and adaptation of protected works. Moral rights protect non-economic interests of the creator like being credited for their work. \parencite{WikiCopy}

According to the Berne Convention copyright is obtained automatically with the act of creation and usually covers a very long time span. \parencite{WikiBerne} There is no registration needed before one can claim and enforce copyright over their own works. Often voluntary registration is available nonetheless to help solve disputes over ownership. Such registrations can help facilitate financial transactions related to a copyright and assignment or transfer of rights to other parties. There exist a number of organizations offering collective management of copyrights, exercising those rights and related transactions on behalf of the rights owner.

\subsection{Copyright on the internet}
Many users of the internet have a number of misconceptions about copyright and how it applies to their usage of internet services. The most important misconceptions are outlined in an article by Vorys at al. \parencite{Vorys2013} and summarized here.

Probably the most common misunderstanding is that everything found on the internet can be freely downloaded and used as the basis of own works. However this is not true in most cases. As mentioned in chapter \ref{sec:CopyDef} copyright is applied automatically to all eligible works and unlicensed reuse is prohibited unless explicitly authorized by the rights owner. The source of works and whether you payed for it or not -- for instance an iTunes purchase versus a YouTube download -- is not relevant.

Another very common misconception often seen on YouTube is that it suffices to either give credit to the original creator, add a copyright notice or claim so called fair use. \parencite{Vorys2013} Adding a copyright notice or crediting the original author has no legal meaning by itself. Proper authorization to use the work is still required. An exception to this can be works published under a Creative Commons license as described in chapter \ref{sec:CC}.

The aforementioned fair use is also a source for confusion among internet users. Fair use is a doctrine that under certain circumstances allows the use of copyrighted material without acquiring permission from the rights holder. It is intended solely as a defence in a lawsuit and can not be regarded as a safe harbour. Whether fair use is applicable has to be decided by a court on a case by case basis (\cite{WikiFair} and \cite{Vorys2013}). As such claiming the usage of copyrighted material as fair use -- for instance in the description text of a YouTube video -- is meaningless.

The legal practice around fair use could however change in the future due to a ruling of the United States Court of Appeals for the Ninth Circuit from 2015 (affirming an earlier decision by the US District Court for the Northern District of California from 2008) in the case Lenz vs. Universal Music Corp. In this case the court reinforced that \textquotedblleft the fair use of a copyrighted work ... is not an infringement of copyright\textquotedblright\ and decided that \textquotedblleft in order for a copyright owner to proceed under the DMCA ... the owner must evaluate whether the material makes fair use of the copyright\textquotedblright\ before sending a take down notice. (\cite{LenzUniversal} and \cite{WikiLenzUniversal})

\section{Patents}
\label{sec:Patents}
According to Wikipedia, a patent is "a set of exclusive rights granted by a sovereign state to an inventor or assignee for a limited period of time in exchange for detailed public disclosure of an invention. [...]  Patents are a form of intellectual property." \parencite{WikiPatent2017}

Patents are thus used to encourage disclosure of new invention - thus making it available for others to spur further developments while at the same time granting the patent-holder privileges to cash in on the invention. The idea behind it can e.g. be found in the Copyright and Patent clause, Article I, Section 8, Clause 8 of the United States Constitution: "The Congress shall have Power [...] To promote the Progress of Science and useful Arts, by securing for limited Times to Authors and Inventors the exclusive Right to their respective Writings and Discoveries [...]" \parencite{Washington1787}.

It is important to note that the patent holder was never intended to be the sole profiteer of the patent deal, instead creation of patents was intended to be a "quid pro quo" deal. On the one hand, the patent holder got a limited monopoly while the public, on the other hand, got the invention in a way that would make it usable and, after the monopoly has ended, reproducible. This was established in various court opinions and orders, even as early as 1858, when Judge Daniels wrote in Kendall v. Winsor (62 .S. 322): 

"It is undeniably true, that the limited and temporary monopoly granted to inventors was never designed for their exclusive profit or advantage; the benefit to the public or community at large was another and doubtless the primary object in granting and securing that monopoly." \parencite{Curtis1858}.

One should take note, that in his opinion, the primary goal was the benefit to the public, not the benefit of the patent holder. Another court opinion defined the exact benefit to the public in 1945 in SCOTT PAPER CO. v. MARCALUS MFG. CO.:

"By the patent laws Congress has given to the inventor opportunity to secure the material rewards for his invention for a limited time, on condition that he make full disclosure for the benefit of the public of the manner of making and using the invention, and that upon the expiration of the patent the public be left free to use the invention." \parencite{Stone1945} Justice Stone thus stressed that patents are merely a compensation for making and using the invention. This goes diametrically against so called NPE, Non-Practising Entities - or, in other words, companies buying or creating patents without plans on using these patents in their products, but abusing them to get license payments from practising entities wanting to incorporate the same ideas in their own products.

\section{Trademark}
\label{sec:Trademark}
In legal terms, a trademark is globally accepted as \textquotedblleft a protected sign which is used to distinguish the products or services of one business from another\textquotedblright \parencite{SwissFederalInstituteofIntellectualProperty2017}.
Commonly, a trademark concerns the logo of a business, but it can also be a slogan or an animation (for example the roaring MGM lion).
Trademarks in itself are not a protected commodity just by their existence (they are in the USA; and are protected by common law, but a party has a stronger case when defending a registered trademark, and may use the \textregistered). They are only protected after registering the trademark, otherwise an opposing party using the same trademark has a much stronger claim to the same mark. \parencite{StatesPatent2016}

\section{Notable Organisations}
\label{sec:Orgas}

\subsection{BSA}
\label{ssec:BSA}
The Software Alliance (changed its name from \underline{B}usiness \underline{S}oftware \underline{A}lliance) is an advocate group representing the interests of the software industry in countries around the world. It was established in 1988 and has its headquarters in Washington DC. Notable members include, amongst others, Adobe, Apple, Autodesk, IBM, Microsoft, Oracle, and Siemens. The current CEO, Victoria Espinel, studied to become a telecommunications lawyer before being appointed as advisor on intellectual property issues to President Obama in 2009 and her subsequent appointment to her current position inside the BSA. \parencite{Rogers2017}
\vspace{\baselineskip}

\noindent The BSA's mission is based on three overriding policies:
\begin{itemize}
	\item Safeguarding intellectual property rights and protections
	\item Opening global markets to digital trade
	\item Enabling the growth of cloud 
\end{itemize}
\parencite{BSA2017}


To reach this stated goal, the Software Alliance believes that patents, copyrights, trademarks, and other IP rights should be comprehensive and enforceable. Trading partners, whether on a global or local scale, should not use their legislation to favour local industries or discourage international competition. And lastly, companies should not abuse their intellectual property to create unfair competitive advantage.

The BSA concentrated its effort for pushing the cloud in pushing for specific policies advancing the cloud, forging bilateral or multilateral agreements that ease the movement of data across borders. It promotes the use of sound data protection practises to expand and fortify the public trust in digital commerce. But even these data protection policies should leave enough leeway for technological innovation and new services, especially cloud computing.

The BSA garnered critic for aggressively enforcing the licence rights of its members under the guise of "audits" and providing bounties for people sending in hints of unlicensed software use in companies. Due to high penalties for unlicensed software and the need for very extensive proof of ownership, many small and medium sized businesses failed to provide ample documentation and were subsequently fined; some of these businesses even to the point of bankruptcy. These actions garnered the BSA the label to be akin to a bully and not the shining knight for IT innovation as presented in their mission statement. \parencite{Gaskin2009}

\subsection{SIIA}
\label{ssec:SIIA}
Similar to the BSA, the Software and Information Industry Association (SIIA) is a trade-group advocating and defending the rights and interests of big software companies. Notable members include, amongst many others, Apple, Facebook, Goldman Sachs and even Credit Suisse. The SIIA has the stated goal of promoting, protecting and informing the software and digital industry. Also similar to the BSA, it has an anti-piracy reward program offering up to 1'000'000\$. \parencite{SIIAreward}

Critics accuse the SIIA to use the complex and confusing IP laws to extort money from small- and medium sized businesses in what is essentially a shakedown. \parencite{Melymuka2006}

\subsection{EFF}
\label{sec:EFF}
The Electronic Frontier Foundation (EFF) is a non-profit organisation and was founded in 1990 as a response to a series of raids performed by the United States Secret Service. These raids were done to track an illegally distributed manual for the emergency call system. In the course of these raids the business of a Mr. Steve Jackson was nearly driven to ruin, as it had its computer equipment confiscated, searched and tampered with. Mr. Jackson turned to civil rights groups to file a lawsuit against the Secret Service, but none of the contacted ones knew enough about the issue. Hearing about the raid and subsequent damage to Mr. Jackson's business, the three founding fathers (John Perry Barlow, John Gilmore and Mitch Kapor) announced the creation of the EFF and their intent to represent Mr. Jackson in court.
The EFF continued to file lawsuits and represent individuals in the upcoming decades. Their focus concern privacy rights, anti censorship, free speech and copyright issues. \parencite{EFFhistory} \parencite{Schultz}

\subsection{CC}
\label{sec:CC}
The Creative Commons (CC) is a non-profit organisation founded by Lawrence Lessig (a Stanford law professor) with the help of the Center for the Public Domain. Upon founding, CC released a set of licences; using these, rights holders can release works to the public with various degrees of freedoms and opportunities for derivative works or permission to use this work.
In 2005 the CC launched the open science initiative with the goal to provide the chance to collaborate and contribute. In this program, research data, lab notes and other research material are freely available and can be reused, redistributed and reproduced.
The present CC has grown to be an international network of partner organisations with like-minded goals and motivations. \parencite{Plotkin2002} \parencite{CC2016}
Critics accuse the CC to be too industry friendly and not doing enough for the common creativity, as the organisation states. \parencite{Berry2005}

\chapter{Current situation}
\label{ch:CurrSit}

\section{Business Models}
\label{sec:BusMods}

\subsection{Open Source Software}
\label{ssec:OSS}
According to Wikipedia, Open Source Software (furthermore abbreviated as OSS) is " computer software with its source code made available with a license in which the copyright holder provides the rights to study, change, and distribute the software to anyone and for any purpose."\parencite{WikiOSS}. One of the founding pillars and most cited sources regarding the motivation for OSS is the GNU Manifesto of Richard Stallmann, written in 1985 wherein he states among other things that "I consider that the Golden Rule requires that if I like a program I must share it with other people who like it. [...] I refuse to break solidarity with other users [...]." (citation GNU Manifesto https://www.gnu.org/gnu/manifesto.en.html).

The idea was later formalized into the four freedoms that make software free software as postulated by the Free Software Foundation: "When you speak about Free Software, you speak about freedom. And more precisely, about the four freedoms to use, study, share and improve the software." (citation needed https://fsfe.org/freesoftware/basics/4freedoms.en.html).

This Idea is in stark contrast to proprietary (closed source) software. One might conclude that OSS thus is a hobby, not a business, as you can't sell the same software anyone can downlaod and compile. This, however, is not true and was already anticipated at the time of the GNU manifesto, which states: "If people would rather pay for GNU plus service than get GNU free without service, a company to provide just service to people who have obtained GNU free ought to be profitable." (citation GNU Manifesto https://www.gnu.org/gnu/manifesto.en.html). The business model around OSS can thus be characterized as selling services to customers. There are several services that can be provided for a charge, e.g.:
\begin{itemize}
	\item Support
	\item Enhancements
	\item Customisations
	\item Education 
\end{itemize}

The question is how one can credibly offer good services and be competitive when everyone else could do the same? It comes down to reputation. If someone has contributed to the programming of a software, this person (legal or natural) gets recognised for that effort (see also https://en.wikipedia.org/wiki/Open-source\_software\_movement). This recognition of the effort and the knowledge of a system can be used to bring leverage to the marketing efforts for selling services around the software. Thus, it is not surprising that big corporations have stepped in to contribute to OSS as the example of Linux, a free operating system, clearly shows. In their annual announcement of the Linux Development Report 2015, the Linux Foundation mentions "The Top 10 organizations sponsoring Linux kernel development since the last report include Intel, Red Hat, Linaro, Samsung, IBM, SUSE, Texas Instruments, Vision Engraving Systems, Google and Renesas." (citation https://www.linuxfoundation.org/news-media/announcements/2015/02/linux-foundation-releases-linux-development-report). While Suse and Redhat are primarily Linux distributors, others use Linux in their products and services. The example of Google, IBM and Samsung illustrate that the interest in OSS not only comes from small and medium-sized enterprises, but also big, global corporations.

The programmers themselves are often paid to work on the OSS, as the same report anouncement also states: "The number of paid developers is on the rise, as companies aggressively recruit top Linux talent. More than 80 percent of kernel development is done by developers who are being paid for their work. Volunteer developers tend not to stay that way for long."


\subsection{Closed Source Software}
\label{ssec:CSS}

\section{Mentality concerning IP rights}
\label{sec:IPMent}

\section{Globalisation \& increased complexity of rights management}
\label{sec:GlobalRightsMgmt}

\chapter{Conclusion}

\newpage

% TODO remove - The bibliography is only printed if there are quotations in the text
\printbibliography

\end{document}          
