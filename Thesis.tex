\documentclass[a4paper]{report}
% Use swiss german letters
\usepackage[utf8]{inputenc}
% Language: english
\usepackage[english]{babel}
% Blindtext package
\usepackage{blindtext}
% Fancy Figures
\usepackage{graphicx}
% Display the Bibliography in the TOC
\usepackage{tocbibind}
% Better lists
\usepackage{enumitem}
% Use biblatex
\usepackage[style=apa,backend=biber,citestyle=authoryear]{biblatex} 
% Define the bibliography file
\addbibresource{bibliography.bib}
% To let LaTeX handle "
\usepackage[autostyle, english = british]{csquotes}
\DeclareLanguageMapping{english}{english-apa}
% To have text wrap around pictures
\usepackage{wrapfig}

% Titlepage
\newcommand*{\titleAP}{\begingroup % Create the command for including the title page in the document
	\centering
	\vspace*{\baselineskip} % Whitespace at the top of the page
	
	{\Large Pascal Baumann} and {\Large Nathanael Gogniat} and {\Large Reto Scheidegger}\\[0.167\textheight] % Author name
	
	{\Huge\bfseries Intellectual property in the digital age}\\[\baselineskip]
	
	{\Large \textit{Thesis paper InfKol FS2017}}\\
	\today
	
	\vspace*{3\baselineskip} % Whitespace at the bottom of the page
	\endgroup}

\graphicspath{{./img/}}

\begin{document}

\titleAP

\begin{abstract}
	\blindtext
\end{abstract}
\newpage

\tableofcontents

\newpage

\chapter{Introduction}
% TODO remove - Labels are used for in book referencing - ie. if you want to reference this chapter you would use \ref{ch:Intro}
\label{ch:Intro}

\section{Copyright}
\label{sec:Copyright}

\section{Patents}
\label{sec:Patents}

\section{Trademark}
\label{sec:Trademark}

\section{Organisations}
\label{sec:Orgas}

\section{Digital Age}
\label{sec:Digital age}
The term "Digital Age" is a description of the current period. The Cambridge Dictionary describes it as "the present time, when most information is in a digital form, especially when compared to the time when computers were not used" (citation needed for http://dictionary.cambridge.org/de/worterbuch/englisch/digital-age visited on 02.03.2017).
 
The definition found in Wikipedia goes a bit further by incorporating the business impacts and the article starts as following: "The Information Age (also known as the Computer Age, Digital Age, or New Media Age) is a period in human history characterized by the shift from traditional industry that the Industrial Revolution brought through industrialization, to an economy based on information computerization" (citation needed for https://en.wikipedia.org/wiki/Information\_Age visited on 02.03.2017).
   
\subsection{Influence of the Digital Age on IP}
The digital availability and the instant accessibility of the information (anytime from anywhere) is in stark contrast to the availability of information in earlier ages. This becomes obvious when looking at the example of books in Europe.

During the period of antiquity, information was written manually on tablets, scrolls and similiar media. Thus, saving information was a time consuming and burdensome task. As a consequence, these tablets and scrolls were rare and expensive. Later on, books with pages of parchment became the norm. These books were copied manually by monks or other scholars. Later on, the single pages of a book were distributed to different copyists to increase the speed of copying. Although this parallelisation of copying made copying a book faster, the effort used was only distributed and not minimised; thus, books remained rare and expensive. This began to change after Johannes Gutenberg invented the printing press with movable type around 1450.




\chapter{Conclusion}

\newpage

% TODO remove - The bibliography is only printed if there are quotations in the text
\printbibliography

\end{document}          
