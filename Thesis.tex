\documentclass[a4paper]{report}
% Use swiss german letters
\usepackage[utf8]{inputenc}
% Language: english
\usepackage[english]{babel}
% Blindtext package
\usepackage{blindtext}
% Fancy Figures
\usepackage{graphicx}
% Display the Bibliography in the TOC
\usepackage{tocbibind}
% Better lists
\usepackage{enumitem}
% Use biblatex
\usepackage[style=apa,backend=biber,citestyle=authoryear]{biblatex} 
% Define the bibliography file
\addbibresource{bibliography.bib}
% To let LaTeX handle "
\usepackage[autostyle, english = british]{csquotes}
\DeclareLanguageMapping{english}{english-apa}
% To have text wrap around pictures
\usepackage{wrapfig}

% Titlepage
\newcommand*{\titleAP}{\begingroup % Create the command for including the title page in the document
	\centering
	\vspace*{\baselineskip} % Whitespace at the top of the page
	
	{\Large Pascal Baumann} and {\Large Nathanael Gogniat} and {\Large Reto Scheidegger}\\[0.167\textheight] % Author name
	
	{\Huge\bfseries Intellectual property in the digital age}\\[\baselineskip]
	
	{\Large \textit{Thesis paper InfKol FS2017}}\\
	\today
	
	\vspace*{3\baselineskip} % Whitespace at the bottom of the page
	\endgroup}

\graphicspath{{./img/}}

\begin{document}

\titleAP

\begin{abstract}
	\blindtext
\end{abstract}
\newpage

\tableofcontents

\newpage

\chapter{Introduction}
% TODO remove - Labels are used for in book referencing - ie. if you want to reference this chapter you would use \ref{ch:Intro}
\label{ch:Intro}

\section{Copyright}
\label{sec:Copyright}

\section{Patents}
\label{sec:Patents}

\section{Trademark}
\label{sec:Trademark}

\section{Notable Organisations}
\label{sec:Orgas}

\subsection{BSA}
\label{ssec:BSA}
The Software Alliance (changed its name from \underline{B}usiness \underline{S}oftware \underline{A}lliance) is an advocate group representing the interests of the software industry in countries around the world. It was established in 1988 and has its headquarters in Washington DC. Notable members include, amongst others, Adobe, Apple, Autodesk, IBM, Microsoft, Oracle, and Siemens. The current CEO, Victoria Espinel, studied to become a telecommunications lawyer before being appointed as advisor on intellectual property issues to President Obama in 2009 and her subsequent appointment to her current position inside the BSA. \parencite{Rogers2017}
\vspace{\baselineskip}

\noindent The BSA's mission is based on three overriding policies:
\begin{itemize}
	\item Safeguarding intellectual property rights and protections
	\item Opening global markets to digital trade
	\item Enabling the growth of cloud 
\end{itemize}
\parencite{BSA2017}


To reach this stated goal, the Software Alliance believes that patents, copyrights, trademarks, and other IP rights should be comprehensive and enforceable. Trading partners, whether on a global or local scale, should not use their legislation to favour local industries or discourage international competition. And lastly, companies should not abuse their intellectual property to create unfair competitive advantage.

The BSA concentrated its effort for pushing the cloud in pushing for specific policies advancing the cloud, forging bilateral or multilateral agreements that ease the movement of data across borders. It promotes the use of sound data protection practices to expand and fortify the public trust in digital commerce. But even these data protection policies should leave enough leeway for technological innovation and new services, especially cloud computing.

The BSA garnered critic for aggressively enforcing the licence rights of its members under the guise of "audits" and providing bounties for people providing hints of unlicensed software use in companies. Due to high penalties for unlicensed software and the need for very extensive proof of ownership, many small and medium sized businesses failed to provide ample documentation and were subsequently fined, some of these businesses even to the point of bankruptcy. These actions garnered the BSA the label to be akin to a bully and not the shining knight for IT innovation as presented in their mission statement. \parencite{Gaskin2009}

\subsection{SIIA}
\label{ssec:SIIA}
Similar to the BSA, the Software and Information Industry Association (SIIA) is a trade-group advocating and defending the rights and interests of big software companies. Notable members include, amongst many others, Apple, Facebook, Goldman Sachs and even Credit Suisse. The SIIA has the stated goal of promoting, protecting and informing the software and digital industry. Also similar to the BSA, it has an anti-piracy reward program offering up to 1'000'000\$. \parencite{SIIAreward}

Critics accuse the SIIA to use the complex and confusing IP laws to extort money from small- and medium sized businesses in what is essentially a shakedown. \parencite{Melymuka2006}

\subsection{EFF}
\label{sec:EFF}
The Electronic Frontier Foundation (EFF) is a non-profit organisation and was founded in 1990 as a response to a series of raids performed by the United States Secret Service. These raids were done to track an illegally distributed manual for the emergency call system. In the course of these raids the business of a Mr. Steve Jackson was nearly driven to ruin, as it had its computer equipment confiscated, searched and tampered with. Mr. Jackson turned to civil rights groups to file a lawsuit against the Secret Service, but none of the contacted ones knew enough about the issue. Hearing about the raid and subsequent damage to Mr. Jackson's business, the three founding fathers (John Perry Barlow, John Gilmore and Mitch Kapor) announced the creation of the EFF and their intent to represent Mr. Jackson in court.
The EFF continued to file lawsuits and represent individuals in the upcoming decades. Their focus concern privacy rights, anti censorship, free speech and copyright issues. \parencite{EFFhistory} \parencite{Schultz}

\subsection{CC}
\label{sec:CC}
The Creative Commons (CC) is a non-profit organisation founded by Lawrence Lessig (a Stanford law professor) with the help of the Center for the Public Domain. Upon founding, CC released a set of licenses; using these, rights holders can release works to the public with various degrees of freedoms and opportunities for derivative works or permission to use this work.
In 2005 the CC launched the open science initiative with the goal to provide the chance to collaborate and contribute. In this program, research data, lab notes and other research material are freely available and can be reused, redistributed and reproduced.
The present CC has grown to be an international network of partner organisations with likeminded goals and motivations. \parencite{Plotkin2002} \parencite{CC2016}
Critics accuse the CC to be too industry friendly and not doing enough for the common creativity, as the organisation states. \parencite{Berry2005}

\section{Digital Age}
\label{sec:Digital age}
The term "Digital Age" is a description of the current period. The Cambridge Dictionary describes it as "the present time, when most information is in a digital form, especially when compared to the time when computers were not used" (citation needed for http://dictionary.cambridge.org/de/worterbuch/englisch/digital-age visited on 02.03.2017).
 
The definition found in Wikipedia goes a bit further by incorporating the business impact and the article starts as following: "The Information Age (also known as the Computer Age, Digital Age, or New Media Age) is a period in human history characterized by the shift from traditional industry that the Industrial Revolution brought through industrialization, to an economy based on information computerization" (citation needed for https://en.wikipedia.org/wiki/Information\_Age visited on 02.03.2017).
   
\subsection{Influence of the Digital Age on IP}
The digital availability and the instant accessibility of the information (anytime from anywhere) is in stark contrast to the availability of information in earlier ages. This becomes obvious when looking at the example of books as means of information distribution in Europe.

During the period of antiquity, information was written manually on tablets, scrolls and similiar media. Thus, saving information was a time consuming and burdensome task.  Later on, books with pages of parchment became the norm. These books were copied manually by monks or other scholars. As a consequence, these items were rare and expensive. Later on, the single pages of a book were distributed to different copyists to increase the speed of copying. Although this parallelisation of copying made copying a book faster, the effort used was only distributed and not minimised; thus, books remained rare and expensive. This began to change after Johannes Gutenberg invented the printing press with movable type around 1450. This development sped up the process of copying books, because the same page could be printed multiple times quickly after the set up of the print plate was completed. However, copying still required the presence of a physical specimen to copy from. With an increasing number of now affordable books, the knowledge contained could slowly spread further, but was still hindered by long, slow and dangerous transports. While the advancement in printing and transportation sped up the process of copying and transportation later on, instant access to information was still not available because the information was bound to a certain geographical location. Before the rise of the internet, people still had to go to libraries for research purposes, oftentime being required to order specimens of the books from other libraries to the closest one - or order a personal copy of a book.

With the establishment of the internet as a commodity, available not only from home and the office but from nearly everywhere via mobile devices, the point of origin of a published text no longer matters for the availability everywhere. The distribution of the information no longer happens in concentric regions around a physical specimen but can become available instantly everywhere at the same time. Getting a copy of a book no longer requires burdensome printing and transport but can be done easily doing some mouse clicks. While this can be beneficial to the human race as a whole, the originator of the books (now in digital form) is no longer sure that he will be compensated, as a book may be distributed using file sharing for free as soon as one copy has been sold.



\chapter{Conclusion}

\newpage

% TODO remove - The bibliography is only printed if there are quotations in the text
\printbibliography

\end{document}          
